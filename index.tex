% Options for packages loaded elsewhere
\PassOptionsToPackage{unicode}{hyperref}
\PassOptionsToPackage{hyphens}{url}
\PassOptionsToPackage{dvipsnames,svgnames,x11names}{xcolor}
%
\documentclass[
  letterpaper,
  DIV=11,
  numbers=noendperiod]{scrreprt}

\usepackage{amsmath,amssymb}
\usepackage{iftex}
\ifPDFTeX
  \usepackage[T1]{fontenc}
  \usepackage[utf8]{inputenc}
  \usepackage{textcomp} % provide euro and other symbols
\else % if luatex or xetex
  \usepackage{unicode-math}
  \defaultfontfeatures{Scale=MatchLowercase}
  \defaultfontfeatures[\rmfamily]{Ligatures=TeX,Scale=1}
\fi
\usepackage{lmodern}
\ifPDFTeX\else  
    % xetex/luatex font selection
\fi
% Use upquote if available, for straight quotes in verbatim environments
\IfFileExists{upquote.sty}{\usepackage{upquote}}{}
\IfFileExists{microtype.sty}{% use microtype if available
  \usepackage[]{microtype}
  \UseMicrotypeSet[protrusion]{basicmath} % disable protrusion for tt fonts
}{}
\makeatletter
\@ifundefined{KOMAClassName}{% if non-KOMA class
  \IfFileExists{parskip.sty}{%
    \usepackage{parskip}
  }{% else
    \setlength{\parindent}{0pt}
    \setlength{\parskip}{6pt plus 2pt minus 1pt}}
}{% if KOMA class
  \KOMAoptions{parskip=half}}
\makeatother
\usepackage{xcolor}
\setlength{\emergencystretch}{3em} % prevent overfull lines
\setcounter{secnumdepth}{-\maxdimen} % remove section numbering
% Make \paragraph and \subparagraph free-standing
\ifx\paragraph\undefined\else
  \let\oldparagraph\paragraph
  \renewcommand{\paragraph}[1]{\oldparagraph{#1}\mbox{}}
\fi
\ifx\subparagraph\undefined\else
  \let\oldsubparagraph\subparagraph
  \renewcommand{\subparagraph}[1]{\oldsubparagraph{#1}\mbox{}}
\fi


\providecommand{\tightlist}{%
  \setlength{\itemsep}{0pt}\setlength{\parskip}{0pt}}\usepackage{longtable,booktabs,array}
\usepackage{calc} % for calculating minipage widths
% Correct order of tables after \paragraph or \subparagraph
\usepackage{etoolbox}
\makeatletter
\patchcmd\longtable{\par}{\if@noskipsec\mbox{}\fi\par}{}{}
\makeatother
% Allow footnotes in longtable head/foot
\IfFileExists{footnotehyper.sty}{\usepackage{footnotehyper}}{\usepackage{footnote}}
\makesavenoteenv{longtable}
\usepackage{graphicx}
\makeatletter
\def\maxwidth{\ifdim\Gin@nat@width>\linewidth\linewidth\else\Gin@nat@width\fi}
\def\maxheight{\ifdim\Gin@nat@height>\textheight\textheight\else\Gin@nat@height\fi}
\makeatother
% Scale images if necessary, so that they will not overflow the page
% margins by default, and it is still possible to overwrite the defaults
% using explicit options in \includegraphics[width, height, ...]{}
\setkeys{Gin}{width=\maxwidth,height=\maxheight,keepaspectratio}
% Set default figure placement to htbp
\makeatletter
\def\fps@figure{htbp}
\makeatother

\KOMAoption{captions}{tableheading}
\makeatletter
\makeatother
\makeatletter
\makeatother
\makeatletter
\@ifpackageloaded{caption}{}{\usepackage{caption}}
\AtBeginDocument{%
\ifdefined\contentsname
  \renewcommand*\contentsname{Table of contents}
\else
  \newcommand\contentsname{Table of contents}
\fi
\ifdefined\listfigurename
  \renewcommand*\listfigurename{List of Figures}
\else
  \newcommand\listfigurename{List of Figures}
\fi
\ifdefined\listtablename
  \renewcommand*\listtablename{List of Tables}
\else
  \newcommand\listtablename{List of Tables}
\fi
\ifdefined\figurename
  \renewcommand*\figurename{Figure}
\else
  \newcommand\figurename{Figure}
\fi
\ifdefined\tablename
  \renewcommand*\tablename{Table}
\else
  \newcommand\tablename{Table}
\fi
}
\@ifpackageloaded{float}{}{\usepackage{float}}
\floatstyle{ruled}
\@ifundefined{c@chapter}{\newfloat{codelisting}{h}{lop}}{\newfloat{codelisting}{h}{lop}[chapter]}
\floatname{codelisting}{Listing}
\newcommand*\listoflistings{\listof{codelisting}{List of Listings}}
\makeatother
\makeatletter
\@ifpackageloaded{caption}{}{\usepackage{caption}}
\@ifpackageloaded{subcaption}{}{\usepackage{subcaption}}
\makeatother
\makeatletter
\@ifpackageloaded{tcolorbox}{}{\usepackage[skins,breakable]{tcolorbox}}
\makeatother
\makeatletter
\@ifundefined{shadecolor}{\definecolor{shadecolor}{rgb}{.97, .97, .97}}
\makeatother
\makeatletter
\makeatother
\makeatletter
\makeatother
\ifLuaTeX
  \usepackage{selnolig}  % disable illegal ligatures
\fi
\IfFileExists{bookmark.sty}{\usepackage{bookmark}}{\usepackage{hyperref}}
\IfFileExists{xurl.sty}{\usepackage{xurl}}{} % add URL line breaks if available
\urlstyle{same} % disable monospaced font for URLs
\hypersetup{
  colorlinks=true,
  linkcolor={blue},
  filecolor={Maroon},
  citecolor={Blue},
  urlcolor={Blue},
  pdfcreator={LaTeX via pandoc}}

\author{}
\date{}

\begin{document}
\ifdefined\Shaded\renewenvironment{Shaded}{\begin{tcolorbox}[breakable, boxrule=0pt, frame hidden, interior hidden, sharp corners, enhanced, borderline west={3pt}{0pt}{shadecolor}]}{\end{tcolorbox}}\fi

\hypertarget{our-vision-of-open-science}{%
\section*{Our vision of Open Science}\label{our-vision-of-open-science}}
\addcontentsline{toc}{section}{Our vision of Open Science}

The core mission of ICArEHB is to build an integrative understanding of
human behavior in prehistoric times and assure that the main outputs of
this research are new and outstanding knowledge, ideas, and
understanding in the fields of prehistoric archaeology and human
evolution, that can be shared with other scientists and with the
society.

Particularly, the destructive nature of archaeology means that we must
stick to the highest standards; we must leave a maximum of elements
available for other scientists and the future generation of
archaeologists. To keep these standards, ICArEHB embraces the concept of
Open Science, i.e., research that is transparent and allows
reproducibility, the best guarantee to stay at the forefront. It means
that, as much as possible, our research projects are validated by our
peers before we start it, our procedures, software, and code, are made
accessible to others, and our results are shared with various audiences
(e.g., other scientists, the general public, policymakers) and our data
are archived for the future.

\hypertarget{open-methodologies}{%
\subsection{Open methodologies}\label{open-methodologies}}

Did you know that even before it starts, projects can be examined and
validated by other experts? This is the case for all projects funded in
ICArEHB, which go through peer review either by the third-party funder
or by an internal review process.

In addition, we encourage our researchers to pre-register their study in
journals and platforms allowing this format. This ensures that our study
plans are robust and helps to reduce the bias of publication, so all
results are shared, even if less exciting than expected.

\hypertarget{open-source}{%
\subsection{Open source}\label{open-source}}

As much as possible, we use open-source software and file formats that
everyone can use. When we write code, or new software and apps, we
publish them and assure that they are available for the rest of the
community. Check out some of these examples here and here.

\hypertarget{open-publications}{%
\subsection{Open publications}\label{open-publications}}

Our scientific outputs are published in the form of articles,
monographs, handbooks, etc., privileging those platforms that will make
our research most visible to our peers, the scientific community, and
the general public.

In ICArEHB, all articles led by our integrated researchers are in Open
Access on the journal website or on the platform of publication. This
means that we prefer to pay to make sure that everyone can have access
to our production instead of asking the readers to pay. You can consult
our list of publications and respective links here.~

\hypertarget{open-communication}{%
\subsection{Open communication}\label{open-communication}}

Scientific publication is not a piece of cake, especially if you are not
used to it. In ICArEHB, we make a special effort to share our results
and our conclusions with other parts of society, with adapted media. The
Science Communication Lab is dedicated to inventing new formats and
producing content for our outreach actions.~

Check here our past and future events of outreach. If you would like
ICArEHB to participate in school activities or in a fair, please contact
us to discuss possibilities.

We exchange updates on good practices in archaeology through our
collaboration with the company of contractual archaeology ERA
Arqueologia.

We also make special efforts to transform scientific work into policy
when necessary.

\hypertarget{open-data}{%
\subsection{Open data}\label{open-data}}

Publications and communications are just the tip of the iceberg when it
comes to research, and the text, tables, and images that we publish are
just elements of a frontpage advertisement to the many hours of
recording, processing, and analysis of what is one of the most valuable
ICArEHB's contributions to the world: our data!

Factual data is the cornerstone of science, and access to data is
crucial in fully understanding and extending the work of others.
Providing free and general access to our data is the most effective way
of ensuring that the fruits of the research can be accessed, read, and
used as the basis for further research.

Bringing together the current demands of funding agencies with our will
to transform the way people share archaeological data, ICArEHB offers a
series of \href{intro.qmd}{guidelines} to researchers, so they can
assess the open science level of a specific study, publication, or
project.

This guide tries to tackle most of the current demands for open science.
It is built around the use of trusted platforms that can support our
researchers throughout their entire project lifecycle, centralizing as
much as possible the different parts of the process, including
Preregistrations, Data Storage and Versioning, Pre-prints, License
attribution, Persistent Identifier Creation, Metadata creation for
maximum machine-actionable Findability, Accessibility, Interoperability,
and Re-usability (FAIR).



\end{document}
